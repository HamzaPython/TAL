
% Default to the notebook output style

    


% Inherit from the specified cell style.




    
\documentclass[11pt]{article}

    
    
    \usepackage[T1]{fontenc}
    % Nicer default font (+ math font) than Computer Modern for most use cases
    \usepackage{mathpazo}

    % Basic figure setup, for now with no caption control since it's done
    % automatically by Pandoc (which extracts ![](path) syntax from Markdown).
    \usepackage{graphicx}
    % We will generate all images so they have a width \maxwidth. This means
    % that they will get their normal width if they fit onto the page, but
    % are scaled down if they would overflow the margins.
    \makeatletter
    \def\maxwidth{\ifdim\Gin@nat@width>\linewidth\linewidth
    \else\Gin@nat@width\fi}
    \makeatother
    \let\Oldincludegraphics\includegraphics
    % Set max figure width to be 80% of text width, for now hardcoded.
    \renewcommand{\includegraphics}[1]{\Oldincludegraphics[width=.8\maxwidth]{#1}}
    % Ensure that by default, figures have no caption (until we provide a
    % proper Figure object with a Caption API and a way to capture that
    % in the conversion process - todo).
    \usepackage{caption}
    \DeclareCaptionLabelFormat{nolabel}{}
    \captionsetup{labelformat=nolabel}

    \usepackage{adjustbox} % Used to constrain images to a maximum size 
    \usepackage{xcolor} % Allow colors to be defined
    \usepackage{enumerate} % Needed for markdown enumerations to work
    \usepackage{geometry} % Used to adjust the document margins
    \usepackage{amsmath} % Equations
    \usepackage{amssymb} % Equations
    \usepackage{textcomp} % defines textquotesingle
    % Hack from http://tex.stackexchange.com/a/47451/13684:
    \AtBeginDocument{%
        \def\PYZsq{\textquotesingle}% Upright quotes in Pygmentized code
    }
    \usepackage{upquote} % Upright quotes for verbatim code
    \usepackage{eurosym} % defines \euro
    \usepackage[mathletters]{ucs} % Extended unicode (utf-8) support
    \usepackage[utf8x]{inputenc} % Allow utf-8 characters in the tex document
    \usepackage{fancyvrb} % verbatim replacement that allows latex
    \usepackage{grffile} % extends the file name processing of package graphics 
                         % to support a larger range 
    % The hyperref package gives us a pdf with properly built
    % internal navigation ('pdf bookmarks' for the table of contents,
    % internal cross-reference links, web links for URLs, etc.)
    \usepackage{hyperref}
    \usepackage{longtable} % longtable support required by pandoc >1.10
    \usepackage{booktabs}  % table support for pandoc > 1.12.2
    \usepackage[inline]{enumitem} % IRkernel/repr support (it uses the enumerate* environment)
    \usepackage[normalem]{ulem} % ulem is needed to support strikethroughs (\sout)
                                % normalem makes italics be italics, not underlines
    

    
    
    % Colors for the hyperref package
    \definecolor{urlcolor}{rgb}{0,.145,.698}
    \definecolor{linkcolor}{rgb}{.71,0.21,0.01}
    \definecolor{citecolor}{rgb}{.12,.54,.11}

    % ANSI colors
    \definecolor{ansi-black}{HTML}{3E424D}
    \definecolor{ansi-black-intense}{HTML}{282C36}
    \definecolor{ansi-red}{HTML}{E75C58}
    \definecolor{ansi-red-intense}{HTML}{B22B31}
    \definecolor{ansi-green}{HTML}{00A250}
    \definecolor{ansi-green-intense}{HTML}{007427}
    \definecolor{ansi-yellow}{HTML}{DDB62B}
    \definecolor{ansi-yellow-intense}{HTML}{B27D12}
    \definecolor{ansi-blue}{HTML}{208FFB}
    \definecolor{ansi-blue-intense}{HTML}{0065CA}
    \definecolor{ansi-magenta}{HTML}{D160C4}
    \definecolor{ansi-magenta-intense}{HTML}{A03196}
    \definecolor{ansi-cyan}{HTML}{60C6C8}
    \definecolor{ansi-cyan-intense}{HTML}{258F8F}
    \definecolor{ansi-white}{HTML}{C5C1B4}
    \definecolor{ansi-white-intense}{HTML}{A1A6B2}

    % commands and environments needed by pandoc snippets
    % extracted from the output of `pandoc -s`
    \providecommand{\tightlist}{%
      \setlength{\itemsep}{0pt}\setlength{\parskip}{0pt}}
    \DefineVerbatimEnvironment{Highlighting}{Verbatim}{commandchars=\\\{\}}
    % Add ',fontsize=\small' for more characters per line
    \newenvironment{Shaded}{}{}
    \newcommand{\KeywordTok}[1]{\textcolor[rgb]{0.00,0.44,0.13}{\textbf{{#1}}}}
    \newcommand{\DataTypeTok}[1]{\textcolor[rgb]{0.56,0.13,0.00}{{#1}}}
    \newcommand{\DecValTok}[1]{\textcolor[rgb]{0.25,0.63,0.44}{{#1}}}
    \newcommand{\BaseNTok}[1]{\textcolor[rgb]{0.25,0.63,0.44}{{#1}}}
    \newcommand{\FloatTok}[1]{\textcolor[rgb]{0.25,0.63,0.44}{{#1}}}
    \newcommand{\CharTok}[1]{\textcolor[rgb]{0.25,0.44,0.63}{{#1}}}
    \newcommand{\StringTok}[1]{\textcolor[rgb]{0.25,0.44,0.63}{{#1}}}
    \newcommand{\CommentTok}[1]{\textcolor[rgb]{0.38,0.63,0.69}{\textit{{#1}}}}
    \newcommand{\OtherTok}[1]{\textcolor[rgb]{0.00,0.44,0.13}{{#1}}}
    \newcommand{\AlertTok}[1]{\textcolor[rgb]{1.00,0.00,0.00}{\textbf{{#1}}}}
    \newcommand{\FunctionTok}[1]{\textcolor[rgb]{0.02,0.16,0.49}{{#1}}}
    \newcommand{\RegionMarkerTok}[1]{{#1}}
    \newcommand{\ErrorTok}[1]{\textcolor[rgb]{1.00,0.00,0.00}{\textbf{{#1}}}}
    \newcommand{\NormalTok}[1]{{#1}}
    
    % Additional commands for more recent versions of Pandoc
    \newcommand{\ConstantTok}[1]{\textcolor[rgb]{0.53,0.00,0.00}{{#1}}}
    \newcommand{\SpecialCharTok}[1]{\textcolor[rgb]{0.25,0.44,0.63}{{#1}}}
    \newcommand{\VerbatimStringTok}[1]{\textcolor[rgb]{0.25,0.44,0.63}{{#1}}}
    \newcommand{\SpecialStringTok}[1]{\textcolor[rgb]{0.73,0.40,0.53}{{#1}}}
    \newcommand{\ImportTok}[1]{{#1}}
    \newcommand{\DocumentationTok}[1]{\textcolor[rgb]{0.73,0.13,0.13}{\textit{{#1}}}}
    \newcommand{\AnnotationTok}[1]{\textcolor[rgb]{0.38,0.63,0.69}{\textbf{\textit{{#1}}}}}
    \newcommand{\CommentVarTok}[1]{\textcolor[rgb]{0.38,0.63,0.69}{\textbf{\textit{{#1}}}}}
    \newcommand{\VariableTok}[1]{\textcolor[rgb]{0.10,0.09,0.49}{{#1}}}
    \newcommand{\ControlFlowTok}[1]{\textcolor[rgb]{0.00,0.44,0.13}{\textbf{{#1}}}}
    \newcommand{\OperatorTok}[1]{\textcolor[rgb]{0.40,0.40,0.40}{{#1}}}
    \newcommand{\BuiltInTok}[1]{{#1}}
    \newcommand{\ExtensionTok}[1]{{#1}}
    \newcommand{\PreprocessorTok}[1]{\textcolor[rgb]{0.74,0.48,0.00}{{#1}}}
    \newcommand{\AttributeTok}[1]{\textcolor[rgb]{0.49,0.56,0.16}{{#1}}}
    \newcommand{\InformationTok}[1]{\textcolor[rgb]{0.38,0.63,0.69}{\textbf{\textit{{#1}}}}}
    \newcommand{\WarningTok}[1]{\textcolor[rgb]{0.38,0.63,0.69}{\textbf{\textit{{#1}}}}}
    
    
    % Define a nice break command that doesn't care if a line doesn't already
    % exist.
    \def\br{\hspace*{\fill} \\* }
    % Math Jax compatability definitions
    \def\gt{>}
    \def\lt{<}
    % Document parameters
    \title{Blue\_score}
    
    
    

    % Pygments definitions
    
\makeatletter
\def\PY@reset{\let\PY@it=\relax \let\PY@bf=\relax%
    \let\PY@ul=\relax \let\PY@tc=\relax%
    \let\PY@bc=\relax \let\PY@ff=\relax}
\def\PY@tok#1{\csname PY@tok@#1\endcsname}
\def\PY@toks#1+{\ifx\relax#1\empty\else%
    \PY@tok{#1}\expandafter\PY@toks\fi}
\def\PY@do#1{\PY@bc{\PY@tc{\PY@ul{%
    \PY@it{\PY@bf{\PY@ff{#1}}}}}}}
\def\PY#1#2{\PY@reset\PY@toks#1+\relax+\PY@do{#2}}

\expandafter\def\csname PY@tok@w\endcsname{\def\PY@tc##1{\textcolor[rgb]{0.73,0.73,0.73}{##1}}}
\expandafter\def\csname PY@tok@c\endcsname{\let\PY@it=\textit\def\PY@tc##1{\textcolor[rgb]{0.25,0.50,0.50}{##1}}}
\expandafter\def\csname PY@tok@cp\endcsname{\def\PY@tc##1{\textcolor[rgb]{0.74,0.48,0.00}{##1}}}
\expandafter\def\csname PY@tok@k\endcsname{\let\PY@bf=\textbf\def\PY@tc##1{\textcolor[rgb]{0.00,0.50,0.00}{##1}}}
\expandafter\def\csname PY@tok@kp\endcsname{\def\PY@tc##1{\textcolor[rgb]{0.00,0.50,0.00}{##1}}}
\expandafter\def\csname PY@tok@kt\endcsname{\def\PY@tc##1{\textcolor[rgb]{0.69,0.00,0.25}{##1}}}
\expandafter\def\csname PY@tok@o\endcsname{\def\PY@tc##1{\textcolor[rgb]{0.40,0.40,0.40}{##1}}}
\expandafter\def\csname PY@tok@ow\endcsname{\let\PY@bf=\textbf\def\PY@tc##1{\textcolor[rgb]{0.67,0.13,1.00}{##1}}}
\expandafter\def\csname PY@tok@nb\endcsname{\def\PY@tc##1{\textcolor[rgb]{0.00,0.50,0.00}{##1}}}
\expandafter\def\csname PY@tok@nf\endcsname{\def\PY@tc##1{\textcolor[rgb]{0.00,0.00,1.00}{##1}}}
\expandafter\def\csname PY@tok@nc\endcsname{\let\PY@bf=\textbf\def\PY@tc##1{\textcolor[rgb]{0.00,0.00,1.00}{##1}}}
\expandafter\def\csname PY@tok@nn\endcsname{\let\PY@bf=\textbf\def\PY@tc##1{\textcolor[rgb]{0.00,0.00,1.00}{##1}}}
\expandafter\def\csname PY@tok@ne\endcsname{\let\PY@bf=\textbf\def\PY@tc##1{\textcolor[rgb]{0.82,0.25,0.23}{##1}}}
\expandafter\def\csname PY@tok@nv\endcsname{\def\PY@tc##1{\textcolor[rgb]{0.10,0.09,0.49}{##1}}}
\expandafter\def\csname PY@tok@no\endcsname{\def\PY@tc##1{\textcolor[rgb]{0.53,0.00,0.00}{##1}}}
\expandafter\def\csname PY@tok@nl\endcsname{\def\PY@tc##1{\textcolor[rgb]{0.63,0.63,0.00}{##1}}}
\expandafter\def\csname PY@tok@ni\endcsname{\let\PY@bf=\textbf\def\PY@tc##1{\textcolor[rgb]{0.60,0.60,0.60}{##1}}}
\expandafter\def\csname PY@tok@na\endcsname{\def\PY@tc##1{\textcolor[rgb]{0.49,0.56,0.16}{##1}}}
\expandafter\def\csname PY@tok@nt\endcsname{\let\PY@bf=\textbf\def\PY@tc##1{\textcolor[rgb]{0.00,0.50,0.00}{##1}}}
\expandafter\def\csname PY@tok@nd\endcsname{\def\PY@tc##1{\textcolor[rgb]{0.67,0.13,1.00}{##1}}}
\expandafter\def\csname PY@tok@s\endcsname{\def\PY@tc##1{\textcolor[rgb]{0.73,0.13,0.13}{##1}}}
\expandafter\def\csname PY@tok@sd\endcsname{\let\PY@it=\textit\def\PY@tc##1{\textcolor[rgb]{0.73,0.13,0.13}{##1}}}
\expandafter\def\csname PY@tok@si\endcsname{\let\PY@bf=\textbf\def\PY@tc##1{\textcolor[rgb]{0.73,0.40,0.53}{##1}}}
\expandafter\def\csname PY@tok@se\endcsname{\let\PY@bf=\textbf\def\PY@tc##1{\textcolor[rgb]{0.73,0.40,0.13}{##1}}}
\expandafter\def\csname PY@tok@sr\endcsname{\def\PY@tc##1{\textcolor[rgb]{0.73,0.40,0.53}{##1}}}
\expandafter\def\csname PY@tok@ss\endcsname{\def\PY@tc##1{\textcolor[rgb]{0.10,0.09,0.49}{##1}}}
\expandafter\def\csname PY@tok@sx\endcsname{\def\PY@tc##1{\textcolor[rgb]{0.00,0.50,0.00}{##1}}}
\expandafter\def\csname PY@tok@m\endcsname{\def\PY@tc##1{\textcolor[rgb]{0.40,0.40,0.40}{##1}}}
\expandafter\def\csname PY@tok@gh\endcsname{\let\PY@bf=\textbf\def\PY@tc##1{\textcolor[rgb]{0.00,0.00,0.50}{##1}}}
\expandafter\def\csname PY@tok@gu\endcsname{\let\PY@bf=\textbf\def\PY@tc##1{\textcolor[rgb]{0.50,0.00,0.50}{##1}}}
\expandafter\def\csname PY@tok@gd\endcsname{\def\PY@tc##1{\textcolor[rgb]{0.63,0.00,0.00}{##1}}}
\expandafter\def\csname PY@tok@gi\endcsname{\def\PY@tc##1{\textcolor[rgb]{0.00,0.63,0.00}{##1}}}
\expandafter\def\csname PY@tok@gr\endcsname{\def\PY@tc##1{\textcolor[rgb]{1.00,0.00,0.00}{##1}}}
\expandafter\def\csname PY@tok@ge\endcsname{\let\PY@it=\textit}
\expandafter\def\csname PY@tok@gs\endcsname{\let\PY@bf=\textbf}
\expandafter\def\csname PY@tok@gp\endcsname{\let\PY@bf=\textbf\def\PY@tc##1{\textcolor[rgb]{0.00,0.00,0.50}{##1}}}
\expandafter\def\csname PY@tok@go\endcsname{\def\PY@tc##1{\textcolor[rgb]{0.53,0.53,0.53}{##1}}}
\expandafter\def\csname PY@tok@gt\endcsname{\def\PY@tc##1{\textcolor[rgb]{0.00,0.27,0.87}{##1}}}
\expandafter\def\csname PY@tok@err\endcsname{\def\PY@bc##1{\setlength{\fboxsep}{0pt}\fcolorbox[rgb]{1.00,0.00,0.00}{1,1,1}{\strut ##1}}}
\expandafter\def\csname PY@tok@kc\endcsname{\let\PY@bf=\textbf\def\PY@tc##1{\textcolor[rgb]{0.00,0.50,0.00}{##1}}}
\expandafter\def\csname PY@tok@kd\endcsname{\let\PY@bf=\textbf\def\PY@tc##1{\textcolor[rgb]{0.00,0.50,0.00}{##1}}}
\expandafter\def\csname PY@tok@kn\endcsname{\let\PY@bf=\textbf\def\PY@tc##1{\textcolor[rgb]{0.00,0.50,0.00}{##1}}}
\expandafter\def\csname PY@tok@kr\endcsname{\let\PY@bf=\textbf\def\PY@tc##1{\textcolor[rgb]{0.00,0.50,0.00}{##1}}}
\expandafter\def\csname PY@tok@bp\endcsname{\def\PY@tc##1{\textcolor[rgb]{0.00,0.50,0.00}{##1}}}
\expandafter\def\csname PY@tok@fm\endcsname{\def\PY@tc##1{\textcolor[rgb]{0.00,0.00,1.00}{##1}}}
\expandafter\def\csname PY@tok@vc\endcsname{\def\PY@tc##1{\textcolor[rgb]{0.10,0.09,0.49}{##1}}}
\expandafter\def\csname PY@tok@vg\endcsname{\def\PY@tc##1{\textcolor[rgb]{0.10,0.09,0.49}{##1}}}
\expandafter\def\csname PY@tok@vi\endcsname{\def\PY@tc##1{\textcolor[rgb]{0.10,0.09,0.49}{##1}}}
\expandafter\def\csname PY@tok@vm\endcsname{\def\PY@tc##1{\textcolor[rgb]{0.10,0.09,0.49}{##1}}}
\expandafter\def\csname PY@tok@sa\endcsname{\def\PY@tc##1{\textcolor[rgb]{0.73,0.13,0.13}{##1}}}
\expandafter\def\csname PY@tok@sb\endcsname{\def\PY@tc##1{\textcolor[rgb]{0.73,0.13,0.13}{##1}}}
\expandafter\def\csname PY@tok@sc\endcsname{\def\PY@tc##1{\textcolor[rgb]{0.73,0.13,0.13}{##1}}}
\expandafter\def\csname PY@tok@dl\endcsname{\def\PY@tc##1{\textcolor[rgb]{0.73,0.13,0.13}{##1}}}
\expandafter\def\csname PY@tok@s2\endcsname{\def\PY@tc##1{\textcolor[rgb]{0.73,0.13,0.13}{##1}}}
\expandafter\def\csname PY@tok@sh\endcsname{\def\PY@tc##1{\textcolor[rgb]{0.73,0.13,0.13}{##1}}}
\expandafter\def\csname PY@tok@s1\endcsname{\def\PY@tc##1{\textcolor[rgb]{0.73,0.13,0.13}{##1}}}
\expandafter\def\csname PY@tok@mb\endcsname{\def\PY@tc##1{\textcolor[rgb]{0.40,0.40,0.40}{##1}}}
\expandafter\def\csname PY@tok@mf\endcsname{\def\PY@tc##1{\textcolor[rgb]{0.40,0.40,0.40}{##1}}}
\expandafter\def\csname PY@tok@mh\endcsname{\def\PY@tc##1{\textcolor[rgb]{0.40,0.40,0.40}{##1}}}
\expandafter\def\csname PY@tok@mi\endcsname{\def\PY@tc##1{\textcolor[rgb]{0.40,0.40,0.40}{##1}}}
\expandafter\def\csname PY@tok@il\endcsname{\def\PY@tc##1{\textcolor[rgb]{0.40,0.40,0.40}{##1}}}
\expandafter\def\csname PY@tok@mo\endcsname{\def\PY@tc##1{\textcolor[rgb]{0.40,0.40,0.40}{##1}}}
\expandafter\def\csname PY@tok@ch\endcsname{\let\PY@it=\textit\def\PY@tc##1{\textcolor[rgb]{0.25,0.50,0.50}{##1}}}
\expandafter\def\csname PY@tok@cm\endcsname{\let\PY@it=\textit\def\PY@tc##1{\textcolor[rgb]{0.25,0.50,0.50}{##1}}}
\expandafter\def\csname PY@tok@cpf\endcsname{\let\PY@it=\textit\def\PY@tc##1{\textcolor[rgb]{0.25,0.50,0.50}{##1}}}
\expandafter\def\csname PY@tok@c1\endcsname{\let\PY@it=\textit\def\PY@tc##1{\textcolor[rgb]{0.25,0.50,0.50}{##1}}}
\expandafter\def\csname PY@tok@cs\endcsname{\let\PY@it=\textit\def\PY@tc##1{\textcolor[rgb]{0.25,0.50,0.50}{##1}}}

\def\PYZbs{\char`\\}
\def\PYZus{\char`\_}
\def\PYZob{\char`\{}
\def\PYZcb{\char`\}}
\def\PYZca{\char`\^}
\def\PYZam{\char`\&}
\def\PYZlt{\char`\<}
\def\PYZgt{\char`\>}
\def\PYZsh{\char`\#}
\def\PYZpc{\char`\%}
\def\PYZdl{\char`\$}
\def\PYZhy{\char`\-}
\def\PYZsq{\char`\'}
\def\PYZdq{\char`\"}
\def\PYZti{\char`\~}
% for compatibility with earlier versions
\def\PYZat{@}
\def\PYZlb{[}
\def\PYZrb{]}
\makeatother


    % Exact colors from NB
    \definecolor{incolor}{rgb}{0.0, 0.0, 0.5}
    \definecolor{outcolor}{rgb}{0.545, 0.0, 0.0}



    
    % Prevent overflowing lines due to hard-to-break entities
    \sloppy 
    % Setup hyperref package
    \hypersetup{
      breaklinks=true,  % so long urls are correctly broken across lines
      colorlinks=true,
      urlcolor=urlcolor,
      linkcolor=linkcolor,
      citecolor=citecolor,
      }
    % Slightly bigger margins than the latex defaults
    
    \geometry{verbose,tmargin=1in,bmargin=1in,lmargin=1in,rmargin=1in}
    
    

    \begin{document}
    
    
    \maketitle
    
    

    
    \section{Bleu score:}\label{bleu-score}

\subsection{Définition:}\label{duxe9finition}

BLEU (the Bilingual Evaluation Understudy), est un algorithme
d'évaluation automatique d'une MT (machine traduction) qui a été proposé
par Kishore Papineni, Salim Roukos, Todd Ward, and Wei-Jing Zhu à 2002,
cet algorithme compare une traduction candidate avec une traduction
référence et calcule un score entre 0 et 1.

\subsection{Approche théorique:}\label{approche-thuxe9orique}

\begin{equation*}
BLEU = BP\prod_{k=1}^{n}{p_k}^{w_k}=e^{ln(BP)}e^{\sum^{n}_{k=1}w_kln(p_k)}
\end{equation*}

\$ Avec: BP = min(1,
\frac{hypothesis\ lenght}{reference\ lenght})\quad et \quad p\_k =
\frac{correct\ k-grams}{total\ k-grams} \$

L'idée principale de BLEU est de comparer les n-grams d'une traduction
candidate avec les n-grams d'une ou plusieurs traductions références et
compter le nombre de sémelarité (calculer une précision pour chaque
k-gram 0\textless{}k\textless{}n). cette méthode respecte les 2 aspects
de traduction: - adequacy: Combien de la signification exprimée dans la
traduction standard (référence) ou la source est également exprimée dans
la traduction candidate. et donc une traduction candidate qui a les
mêmes mots (1-grams) d'une traduction référence servi à respecter cette
norme. - fluency: est-ce qu'une traduction candidate est
grammaticalement bien informée ou non. ce qui peut être mesuré par les
n-grams.

L'algorithme BLEU prit en considération aussi qu'une traduction
candidate ne doit pas être trop longue ni trop court par rapport au
référence. donc: - si elle est trop longue que sa référence, le score
calculé va être pénalisé par les n-grams. - si elle est trop court que
sa référence, le score calculé va être pénalisé par un facteur
multiplicative (BP). donc une bonne traduction doit avoir une longueur,
un ordre des mots, un choix des mots proches de sa référence.

\subsection{Approche pratique:}\label{approche-pratique}

Natural Language Toolkit library (NLTK) propose une implémentation de
BLEU qu'on va utiliser dans notre projet.

    \subsubsection{Initialisation:}\label{initialisation}

    \begin{Verbatim}[commandchars=\\\{\}]
{\color{incolor}In [{\color{incolor}1}]:} \PY{n}{reference} \PY{o}{=} \PY{p}{[}\PY{l+s+s2}{\PYZdq{}}\PY{l+s+s2}{my name is hamza jamal}\PY{l+s+s2}{\PYZdq{}}\PY{o}{.}\PY{n}{split}\PY{p}{(}\PY{p}{)}\PY{p}{]}
        \PY{n}{candidate} \PY{o}{=} \PY{l+s+s2}{\PYZdq{}}\PY{l+s+s2}{hamza jamal is a name of me}\PY{l+s+s2}{\PYZdq{}}\PY{o}{.}\PY{n}{split}\PY{p}{(}\PY{p}{)}
\end{Verbatim}


    \subsubsection{Précision:}\label{pruxe9cision}

\paragraph{Les n-grams:}\label{les-n-grams}

    \begin{Verbatim}[commandchars=\\\{\}]
{\color{incolor}In [{\color{incolor}2}]:} \PY{k+kn}{from} \PY{n+nn}{nltk} \PY{k}{import} \PY{n}{ngrams}
        
        \PY{n}{precisions} \PY{o}{=} \PY{p}{[}\PY{p}{]}
        \PY{n+nb}{print}\PY{p}{(}\PY{l+s+s2}{\PYZdq{}}\PY{l+s+s2}{candidate n\PYZhy{}grams:}\PY{l+s+s2}{\PYZdq{}}\PY{p}{)}
        \PY{k}{for} \PY{n}{i} \PY{o+ow}{in} \PY{n+nb}{range}\PY{p}{(}\PY{l+m+mi}{1}\PY{p}{,}\PY{l+m+mi}{5}\PY{p}{)}\PY{p}{:}
            \PY{n}{cand\PYZus{}i\PYZus{}grams} \PY{o}{=} \PY{n+nb}{list}\PY{p}{(}\PY{n}{ngrams}\PY{p}{(}\PY{n}{candidate}\PY{p}{,} \PY{n}{i}\PY{p}{)}\PY{p}{)}
            \PY{n+nb}{print}\PY{p}{(}\PY{l+s+s2}{\PYZdq{}}\PY{l+s+si}{\PYZob{}\PYZcb{}}\PY{l+s+s2}{\PYZhy{}grams: }\PY{l+s+si}{\PYZob{}\PYZcb{}}\PY{l+s+s2}{\PYZdq{}}\PY{o}{.}\PY{n}{format}\PY{p}{(}\PY{n}{i}\PY{p}{,}\PY{n}{cand\PYZus{}i\PYZus{}grams}\PY{p}{)}\PY{p}{)}
        \PY{n+nb}{print}\PY{p}{(}\PY{p}{)}
        \PY{n+nb}{print}\PY{p}{(}\PY{l+s+s2}{\PYZdq{}}\PY{l+s+s2}{reference n\PYZhy{}grams:}\PY{l+s+s2}{\PYZdq{}}\PY{p}{)}
        \PY{k}{for} \PY{n}{i} \PY{o+ow}{in} \PY{n+nb}{range}\PY{p}{(}\PY{l+m+mi}{1}\PY{p}{,}\PY{l+m+mi}{5}\PY{p}{)}\PY{p}{:}
            \PY{n}{ref\PYZus{}i\PYZus{}grams} \PY{o}{=} \PY{n+nb}{list}\PY{p}{(}\PY{n}{ngrams}\PY{p}{(}\PY{o}{*}\PY{n}{reference}\PY{p}{,} \PY{n}{i}\PY{p}{)}\PY{p}{)}
            \PY{n+nb}{print}\PY{p}{(}\PY{l+s+s2}{\PYZdq{}}\PY{l+s+si}{\PYZob{}\PYZcb{}}\PY{l+s+s2}{\PYZhy{}grams: }\PY{l+s+si}{\PYZob{}\PYZcb{}}\PY{l+s+s2}{\PYZdq{}}\PY{o}{.}\PY{n}{format}\PY{p}{(}\PY{n}{i}\PY{p}{,}\PY{n}{ref\PYZus{}i\PYZus{}grams}\PY{p}{)}\PY{p}{)}
\end{Verbatim}


    \begin{Verbatim}[commandchars=\\\{\}]
candidate n-grams:
1-grams: [('hamza',), ('jamal',), ('is',), ('a',), ('name',), ('of',), ('me',)]
2-grams: [('hamza', 'jamal'), ('jamal', 'is'), ('is', 'a'), ('a', 'name'), ('name', 'of'), ('of', 'me')]
3-grams: [('hamza', 'jamal', 'is'), ('jamal', 'is', 'a'), ('is', 'a', 'name'), ('a', 'name', 'of'), ('name', 'of', 'me')]
4-grams: [('hamza', 'jamal', 'is', 'a'), ('jamal', 'is', 'a', 'name'), ('is', 'a', 'name', 'of'), ('a', 'name', 'of', 'me')]

reference n-grams:
1-grams: [('my',), ('name',), ('is',), ('hamza',), ('jamal',)]
2-grams: [('my', 'name'), ('name', 'is'), ('is', 'hamza'), ('hamza', 'jamal')]
3-grams: [('my', 'name', 'is'), ('name', 'is', 'hamza'), ('is', 'hamza', 'jamal')]
4-grams: [('my', 'name', 'is', 'hamza'), ('name', 'is', 'hamza', 'jamal')]

    \end{Verbatim}

    \subsubsection{Calculer la précision:}\label{calculer-la-pruxe9cision}

    \begin{Verbatim}[commandchars=\\\{\}]
{\color{incolor}In [{\color{incolor}3}]:} \PY{k+kn}{from} \PY{n+nn}{nltk}\PY{n+nn}{.}\PY{n+nn}{translate}\PY{n+nn}{.}\PY{n+nn}{bleu\PYZus{}score} \PY{k}{import} \PY{n}{modified\PYZus{}precision}
        
        \PY{n}{precisions} \PY{o}{=} \PY{p}{[}\PY{p}{]}
        \PY{k}{for} \PY{n}{i} \PY{o+ow}{in} \PY{n+nb}{range}\PY{p}{(}\PY{l+m+mi}{1}\PY{p}{,}\PY{l+m+mi}{5}\PY{p}{)}\PY{p}{:}
            \PY{n}{precision} \PY{o}{=} \PY{n}{modified\PYZus{}precision}\PY{p}{(}\PY{n}{reference}\PY{p}{,} \PY{n}{candidate}\PY{p}{,} \PY{n}{i}\PY{p}{)}
            \PY{n}{precisions}\PY{o}{.}\PY{n}{append}\PY{p}{(}\PY{n}{precision}\PY{p}{)}
            \PY{n+nb}{print}\PY{p}{(}\PY{l+s+s2}{\PYZdq{}}\PY{l+s+s2}{précision de }\PY{l+s+si}{\PYZob{}0\PYZcb{}}\PY{l+s+s2}{\PYZhy{}gram = }\PY{l+s+si}{\PYZob{}1\PYZcb{}}\PY{l+s+s2}{\PYZdq{}}\PY{o}{.}\PY{n}{format}\PY{p}{(}\PY{n}{i}\PY{p}{,}\PY{n}{precision}\PY{p}{)}\PY{p}{)}
\end{Verbatim}


    \begin{Verbatim}[commandchars=\\\{\}]
précision de 1-gram = 4/7
précision de 2-gram = 1/6
précision de 3-gram = 0/5
précision de 4-gram = 0/4

    \end{Verbatim}

    \subsubsection{Facteur de
pénalisation:}\label{facteur-de-puxe9nalisation}

    \begin{Verbatim}[commandchars=\\\{\}]
{\color{incolor}In [{\color{incolor}4}]:} \PY{k+kn}{from} \PY{n+nn}{nltk}\PY{n+nn}{.}\PY{n+nn}{translate}\PY{n+nn}{.}\PY{n+nn}{bleu\PYZus{}score} \PY{k}{import} \PY{n}{brevity\PYZus{}penalty}
        
        \PY{n}{ref\PYZus{}len} \PY{o}{=} \PY{n+nb}{len}\PY{p}{(}\PY{n}{reference}\PY{p}{)}
        \PY{n}{cand\PYZus{}len} \PY{o}{=} \PY{n+nb}{len}\PY{p}{(}\PY{n}{candidate}\PY{p}{)}
        \PY{n}{BP} \PY{o}{=} \PY{n}{brevity\PYZus{}penalty}\PY{p}{(}\PY{n}{ref\PYZus{}len}\PY{p}{,} \PY{n}{cand\PYZus{}len}\PY{p}{)}
        \PY{n+nb}{print}\PY{p}{(}\PY{n}{BP}\PY{p}{)}
\end{Verbatim}


    \begin{Verbatim}[commandchars=\\\{\}]
1

    \end{Verbatim}

    \subsubsection{Calculer Bleu score:}\label{calculer-bleu-score}

    \begin{Verbatim}[commandchars=\\\{\}]
{\color{incolor}In [{\color{incolor}5}]:} \PY{k+kn}{from} \PY{n+nn}{math} \PY{k}{import} \PY{n}{exp}\PY{p}{,} \PY{n}{log}
        \PY{c+c1}{\PYZsh{} exp(log(0))}
        \PY{n}{exp\PYZus{}log\PYZus{}precisions} \PY{o}{=} \PY{n+nb}{map}\PY{p}{(}\PY{k}{lambda} \PY{n}{p\PYZus{}k}\PY{p}{:}\PY{n}{exp}\PY{p}{(}\PY{l+m+mi}{1}\PY{o}{/}\PY{l+m+mi}{4}\PY{o}{*}\PY{n}{log}\PY{p}{(}\PY{n}{p\PYZus{}k}\PY{p}{)}\PY{p}{)}\PY{p}{,} \PY{n}{precisions}\PY{p}{)}
        \PY{n}{BLEU\PYZus{}score}\PY{o}{=}\PY{n}{BP}
        \PY{k}{for} \PY{n}{exp\PYZus{}log\PYZus{}precision} \PY{o+ow}{in} \PY{n}{exp\PYZus{}log\PYZus{}precisions}\PY{p}{:}
            \PY{n}{BLEU\PYZus{}score} \PY{o}{*}\PY{o}{=} \PY{n}{exp\PYZus{}log\PYZus{}precision}
        \PY{n+nb}{print}\PY{p}{(}\PY{n}{BLEU\PYZus{}score}\PY{p}{)}
\end{Verbatim}


    \begin{Verbatim}[commandchars=\\\{\}]

        ---------------------------------------------------------------------------

        ValueError                                Traceback (most recent call last)

        <ipython-input-5-93a0cd86000f> in <module>()
          3 exp\_log\_precisions = map(lambda p\_k:exp(1/4*log(p\_k)), precisions)
          4 BLEU\_score=BP
    ----> 5 for exp\_log\_precision in exp\_log\_precisions:
          6     BLEU\_score *= exp\_log\_precision
          7 print(BLEU\_score)


        <ipython-input-5-93a0cd86000f> in <lambda>(p\_k)
          1 from math import exp, log
          2 \# exp(log(0))
    ----> 3 exp\_log\_precisions = map(lambda p\_k:exp(1/4*log(p\_k)), precisions)
          4 BLEU\_score=BP
          5 for exp\_log\_precision in exp\_log\_precisions:


        ValueError: math domain error

    \end{Verbatim}

    \subsubsection{Exemples:}\label{exemples}

\paragraph{- 1 phrase 1 référence:}\label{phrase-1-ruxe9fuxe9rence}

    \begin{Verbatim}[commandchars=\\\{\}]
{\color{incolor}In [{\color{incolor}6}]:} \PY{k+kn}{from} \PY{n+nn}{nltk}\PY{n+nn}{.}\PY{n+nn}{translate}\PY{n+nn}{.}\PY{n+nn}{bleu\PYZus{}score} \PY{k}{import} \PY{n}{sentence\PYZus{}bleu}
        
        \PY{c+c1}{\PYZsh{} score = sentence\PYZus{}bleu(reference, candidate)}
        \PY{c+c1}{\PYZsh{}\PYZsh{} solutions:}
        \PY{c+c1}{\PYZsh{}1\PYZhy{} préciser le n:}
        \PY{c+c1}{\PYZsh{} score = sentence\PYZus{}bleu(reference, candidate, weights=(0.5,0.5))}
        \PY{c+c1}{\PYZsh{} 2\PYZhy{} utiliser smoothing functions:}
        \PY{c+c1}{\PYZsh{} from nltk.translate.bleu\PYZus{}score import SmoothingFunction}
        \PY{c+c1}{\PYZsh{} my\PYZus{}smoothing\PYZus{}function = SmoothingFunction()}
        \PY{c+c1}{\PYZsh{} score = sentence\PYZus{}bleu(reference, candidate, smoothing\PYZus{}function=my\PYZus{}smoothing\PYZus{}function.method0)}
        \PY{c+c1}{\PYZsh{} 3\PYZhy{} utiliser emulate\PYZus{}multibleu}
        \PY{n}{score} \PY{o}{=} \PY{n}{sentence\PYZus{}bleu}\PY{p}{(}\PY{n}{reference}\PY{p}{,} \PY{n}{candidate}\PY{p}{)}
        
        \PY{n+nb}{print}\PY{p}{(}\PY{n}{score}\PY{p}{)}
\end{Verbatim}


    \begin{Verbatim}[commandchars=\\\{\}]
0.5555238068023582

    \end{Verbatim}

    \begin{Verbatim}[commandchars=\\\{\}]
/home/hamza/anaconda3/lib/python3.6/site-packages/nltk/translate/bleu\_score.py:490: UserWarning: 
Corpus/Sentence contains 0 counts of 3-gram overlaps.
BLEU scores might be undesirable; use SmoothingFunction().
  warnings.warn(\_msg)

    \end{Verbatim}

    \paragraph{- 1 phrase plusieurs
références:}\label{phrase-plusieurs-ruxe9fuxe9rences}

    \begin{Verbatim}[commandchars=\\\{\}]
{\color{incolor}In [{\color{incolor}7}]:} \PY{k+kn}{from} \PY{n+nn}{nltk}\PY{n+nn}{.}\PY{n+nn}{translate}\PY{n+nn}{.}\PY{n+nn}{bleu\PYZus{}score} \PY{k}{import} \PY{n}{sentence\PYZus{}bleu}
        \PY{n}{references} \PY{o}{=} \PY{p}{[}\PY{l+s+s2}{\PYZdq{}}\PY{l+s+s2}{my name is hamza jamal how about you?}\PY{l+s+s2}{\PYZdq{}}\PY{o}{.}\PY{n}{split}\PY{p}{(}\PY{p}{)}\PY{p}{,} \PY{l+s+s2}{\PYZdq{}}\PY{l+s+s2}{hamza jamal is my name and you?}\PY{l+s+s2}{\PYZdq{}}\PY{o}{.}\PY{n}{split}\PY{p}{(}\PY{p}{)}\PY{p}{]}
        \PY{n}{candidate\PYZus{}2} \PY{o}{=} \PY{l+s+s2}{\PYZdq{}}\PY{l+s+s2}{hamza jamal is my name how about you?}\PY{l+s+s2}{\PYZdq{}}\PY{o}{.}\PY{n}{split}\PY{p}{(}\PY{p}{)}
        \PY{n}{score} \PY{o}{=} \PY{n}{sentence\PYZus{}bleu}\PY{p}{(}\PY{n}{references}\PY{p}{,} \PY{n}{candidate\PYZus{}2}\PY{p}{)}
        \PY{n+nb}{print}\PY{p}{(}\PY{n}{score}\PY{p}{)}
\end{Verbatim}


    \begin{Verbatim}[commandchars=\\\{\}]
0.691441569283882

    \end{Verbatim}

    \paragraph{- text (plusieurs phrases) plusieurs
références:}\label{text-plusieurs-phrases-plusieurs-ruxe9fuxe9rences}

    \begin{Verbatim}[commandchars=\\\{\}]
{\color{incolor}In [{\color{incolor} }]:} \PY{k}{pass}
\end{Verbatim}



    % Add a bibliography block to the postdoc
    
    
    
    \end{document}
